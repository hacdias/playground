Este quarto capítulo tem como objetivo mostrar-lhe as formas de controlar o fluxo de uma aplicação, de um algoritmo, de um programa.

\begin{defi}
\textbf{Controlo de Fluxo} refere-se ao controlo que se tem sobre a ordem de comandos a serem executados no decorrer de um programa.
\end{defi}

Ao controlar o fluxo pode-se direcionar o utilizador para as ações que este escolheu e executar apenas trechos de código dependendo de uma determinada condição, controlando a ordem pela qual os comandos são executados.

Existem diversas estruturas que nos permitem controlar o fluxo de uma aplicação. A maioria das que são aqui abordadas são transversais à maioria das linguagens de programação existentes.

Antes de continuar aconselhamos a que reveja os \textbf{operadores relacionais e lógicos} no capítulo 3.

\section{Estrutura \texttt{if/else}}

A primeira estrutura a abordar é a conhecida \texttt{if else} o que, numa tradução literal para Português, quer dizer \quotes{se caso contrário}. Com esta estrutura, um determinado trecho de código pode ser executado dependendo do resultado de um teste lógico.

\begin{defi}
Um \textbf{teste lógico} consiste na determinação da verdade ou falsidade de uma condição.
\end{defi}

\paragraph{Sintaxe}

\begin{lstlisting}
if (condição) {  
    // código a executar caso a condição seja verificada
} else {  
    // caso contrário, isto é executado.  
}  
\end{lstlisting}

\paragraph{Exemplo}

Imagine que necessita criar um pequeno programa que deve imprimir se o valor de uma variável é maior, menor ou igual a 50. Como irá proceder? A criação deste algoritmo é deveras simples, bastando inicializar a variável e efetuar um teste lógico. Veja então como este problema poderia ser resolvido:

\begin{lstlisting}
#include <stdio.h>  
   
int main() {      
    int n = 25;   
   
    if (n >= 50) {  
        printf("O valor %d é maior ou igual a 50.", n);  
    } else {  
        printf("O valor %d é menor que 50.", n);  
    }  
	   
    return 0;  
}  
\end{lstlisting}

Se executar o trecho de código anterior, a mensagem \texttt{O valor 25 é menor que 50.} será imprimida, pois não se verificou a condição \texttt{n >= 50}, executando-se o código dentro do bloco \texttt{else}.

Imagine agora que precisa verificar as seguintes três condições:

\begin{itemize}
\item É maior que 50?
\item É igual a 50?
\item É menor que 50?
\end{itemize}

O caminho mais óbvio seria o seguinte:

\begin{lstlisting}
if (n > 50) {  
    printf("O valor %d é maior que 50.\n", n);  
}   
   
if (n < 50) {  
    printf("O valor %d é menor que 50.\n", n);  
}  
   
if (n == 50) {  
    printf("A variável é igual a 50.\n");  
}  
\end{lstlisting}

O algoritmo acima é um pouco repetitivo e extenso; pode ser simplificado ao ser agregado em apenas um teste sequencial como seguinte:

\begin{lstlisting}
if (n == 50) {  
    printf("A variável é igual a 50.\n"); //A
} else if (n < 50) {  
    printf("O valor %d é menor que 50.\n", n); //B
} else {  
    printf("O valor %d é maior que 50.\n", n); //C  
}  
\end{lstlisting}

Podemos \quotes{traduzir} o código anterior para uma linguagem corrente da seguinte forma: \texttt{Se n for igual a 50; então faz A, ou se n for menor que 50 faz B; caso contrário faz \texttt{C}}.

\section{Estrutura \texttt{while}}

Outra estrutura para controlar o fluxo que é muito importante é a estrutura \texttt{while}, que em Português significa \quotes{enquanto}. Esta estrutura permite repetir um determinado trecho de código enquanto uma condição for verdadeira.

\paragraph{Sintaxe}

\begin{lstlisting}
while(condição) {  
   //Algo acontece  
}  
\end{lstlisting}

Ora vejamos um exemplo: precisa imprimir todos os números entre 0 e 100 (inclusive). Para isso não é necessário utilizar o comando \texttt{printf} 101 vezes, bastando utilizar a repetição \texttt{while}. Ora veja:

\begin{lstlisting}
#include <stdio.h>  

int main() {     
    int num = 0;  
   
    while(num <= 100) { 
    	num++; 
        printf("%d\n", num);
    }  
   
    return 0;  
}
\end{lstlisting}

Pode-se traduzir o trecho anterior para uma linguagem corrente da seguinte forma: \texttt{Enquanto num for menor ou igual a 100, imprime a variável num e incrementa-lhe um valor}.

\section{Estrutura \texttt{switch}}

A estrutura \texttt{switch} está presente na maioria das linguagens de programação, que numa tradução literal para Português significa \quotes{interruptor}. Esta estrutura é deveras útil quando temos que executar uma ação dependendo do valor de uma variável.

A estrutura de controlo \texttt{switch} pode ser utilizada como forma de abreviar um teste lógico \texttt{if else} longo.

\paragraph{Sintaxe}

\begin{lstlisting}
switch(variavel) {
    case "valorUm":
        //Código da operação
        break;
    case "valorDois":
        //Código da operação
        break;

    //...	
	
    default:
        //Código executado caso não seja nenhuma das opções anteriores
        break;
}  
\end{lstlisting}

Imagine que tem que pedir ao utilizador um valor e que vai executar uma ação dependendo o valor que o utilizador escolheu. Esse menu tem 5 opções. Poderia resolver este problema da seguinte forma:

\begin{lstlisting}
#include <stdio.h>
 
int main() {
    int opcao;
 
    printf("Insira a opção:\n");
    scanf("%d", &opcao);
 
    switch(opcao) {
        case 1:
            printf("Escolheu a opção 1");
            break;
        case 2:
            printf("Escolheu a opção 2");
            break;
        case 3:
            printf("Escolheu a opção 3");
            break;
        case 4:
            printf("Escolheu a opção 4");
            break;
        case 5:
            printf("Escolheu a opção 5");
            break;
        default:
            printf("Opção inexistente.");
            break;        
    }
 
    return 0;
}
\end{lstlisting}

O código acima faz: em primeiro lugar, tudo depende do valor da variável \texttt{opcao}. Caso seja 1, será imprimida a mensagem \texttt{Escolheu a opção 1} e por aí a diante. Caso a opção inserida não exista no código, o código contido em \texttt{default} irá ser executado. 

Mais à frente iremos falar mais sobre o \texttt{break}. Por agora não lhe dê muita importância, mas coloque-o sempre.

O algoritmo anteriormente reproduzido pode também tomar a forma de uma sequência de \texttt{if else}, embora com uma menor legibilidade. Ora veja:

\begin{lstlisting}
#include <stdio.h>
 
int main() {
    int option;
 
    printf("Insira a opção:\n");
    scanf("%d", &option);
 
    if (option == 1) {
        printf("Escolheu a opção 1");
    } else if (option == 2) {
        printf("Escolheu a opção 2");
    } else if (option == 3) {
        printf("Escolheu a opção 3");
    } else if (option == 4) {
        printf("Escolheu a opção 4");
    } else if (option == 5) {
        printf("Escolheu a opção 5");
    } else {
        printf("Opção inexistente.");
    }
 
    return 0;
}
\end{lstlisting}

\section{Estrutura \texttt{do/while}}

Outra forma de controlar o fluxo a abordar é a estrutura \texttt{do while}, que tem um nome semelhante à já abordada \texttt{while}. A diferença existente entre essas duas estruturas é pequena, mas importante.

Ao contrário do que acontece na estrutura \texttt{while}, no \texttt{do while}, o código é executado primeiro e só depois é que a condição é testada. Se a condição for verdadeira, o código é executado novamente. Podemos concluir que esta estrutura obriga o código a ser \textbf{executado pelo menos uma vez}.

\paragraph{Sintaxe}

\begin{lstlisting}
do
{
   //código a ser repetido
} while (condição);
\end{lstlisting}

Imagine agora que precisa criar uma pequena calculadora (ainda em linha de comandos) que receba dois número e que, posteriormente, efetua uma soma, subtração, multiplicação ou divisão. Esta calculadora, após cada cálculo deverá pedir ao utilizador para inserir se quer continuar a realizar cálculos ou não. Poderíamos proceder da seguinte forma:

\begin{lstlisting}
#include <stdio.h>
 
int main() {
  int calcular;
 
  do {
 
    char operacao;
    float num1,
        num2;
 
    // limpeza do buffer. ou __fpurge(stdin); em linux
    fflush(stdin);    
 
    printf("Escolha a operação [+ - * / ]: ");
    scanf("%c", &operacao);
 
    printf("Insira o primeiro número: ");
    scanf("%f", &num1);
 
    printf("Insira o segundo número: ");
    scanf("%f", &num2);
 
    switch(operacao) {
      case '+':
        printf("%.2f + %.2f = %.2f\n", num1, num2, num1 + num2);
        break;
      case '-':
        printf("%.2f - %.2f = %.2f\n", num1, num2, num1 - num2);
        break;
      case '*':
        printf("%.2f * %.2f = %.2f\n", num1, num2, num1 * num2);
        break;
      case '/':
        printf("%.2f / %.2f = %.2f\n", num1, num2, num1 / num2);
        break;
      default:
        printf("Digitou uma operação inválida.\n");
        break;
    }
 
    printf("Insira 0 para sair ou 1 para continuar: ");
    scanf("%d", &calcular);
 
  /* Verifica-se o valor da variável calcular. Se for 0 é considerado falso
  e o código não é executado mais vez nenhuma. Caso seja um número diferente
  de 0, a condição retornará um valor que representa true (todos os números
  exceto 0) e continuar-se-à a executar o código. */
  } while (calcular);
 
  return 0;
 
}
\end{lstlisting}

\section{Estrutura \texttt{for}}

A última estrutura de controlo de fluxo a abordar, mas não o menos importante, é a estrutura \texttt{for}. Esta estrutura \textit{loop} é um pouco mais complexa que as anteriores, mas muito útil e permite reutilizar código.

\begin{defi}
O termo inglês \textit{\textbf{loop}} refere-se a todas as estruturas que efetuam repetição: \texttt{while}, \texttt{do while} e \texttt{for}.
\end{defi}

\paragraph{Sintaxe}

\begin{lstlisting}
for(inicio_do_loop ; condição ; termino_de_cada_iteração) {
 //código a ser executado
}
\end{lstlisting}

Onde:

\begin{itemize}
\item \texttt{inicio\_do\_loop} \(\rightarrow\) uma ação que é executada no início do ciclo das repetições;
\item \texttt{condição} \(\rightarrow\) a condição para que o código seja executado;
\item \texttt{termino\_de\_cada\_iteração} \(\rightarrow\) uma ação a executar no final de cada iteração.
\end{itemize}

Imagine agora que precisa de imprimir todos os números pares de 0 a 1000. Para isso poderia recorrer à estrutura \texttt{while} e a uma condição \texttt{if} da seguinte forma:


\begin{lstlisting}
#include <stdio.h>
 
int main() {
    int num = 0;
 
    while (num <= 1000) {
        if (num % 2 = 0) {
            printf("%d\n", num);
        }
        
        num++;
    }
 
    return 0;
}
\end{lstlisting}

Utilizando a estrutura \texttt{for}, o algoritmo acima poderia ser reduzido ao seguinte:

\begin{lstlisting}
#include <stdio.h>
 
int main() {
    for(int num = 0; num <= 1000; num++) {
        if (num % 2 = 0) {
            printf("%d\n", num);
        }
    }

    return 0;
}
\end{lstlisting}

\section{Interrupção do fluxo}

As estruturas de controlo de fluxo são muito importantes, mas aprender como se as interrompe também o é. Por vezes é necessário interromper uma determinada repetição dependendo de um teste lógico interno à repetição.

\subsection{Terminar ciclo com \texttt{break}}

O \texttt{break} permite que interrompamos o fluxo em \textit{loops} e na estrutura \texttt{switch}. Se não colocarmos este comando no final de cada caso da estrutura \texttt{switch}, o código continuaria a ser executado, incluindo as restantes opções.

Imagine que por algum motivo precisa encontrar o primeiro número divisível por 17, 18 e 20 entre 1 e 1000000. Iria, provavelmente, utilizar a estrutura \texttt{for} para percorrer todos estes números. Veja como poderíamos fazer:

\begin{lstlisting}
#include <stdio.h>
 
int main() {      
		int num = 0;
           
        for (int count = 1; count <= 1000000; count++) {
            if(num == 0) {
                if((count % 17 == 0) && (count % 18 == 0) && (count % 20 == 0)) {
                    num = count;
                }
            }
        }
 
        printf("O número divisível por 17, 18 e 20 entre 1 e 1000000 é: %d", num);
        return 0;
}
\end{lstlisting}

Depois de encontrar o número em questão, que é 3060, é necessário continuar a executar o \texttt{loop}? Seriam executadas mais 996940 iterações (repetições). Tal não é necessário e podemos poupar os recursos consumidos, parando o ciclo de repetições. Ora veja:

\begin{lstlisting}
#include <stdio.h>
 
int main() {           
        int num = 0;
        
        for (int count = 1; count <= 1000000; count++) {
            if(num == 0 && (count%17==0) && (count%18==0) && (count%20==0)) {
                num = count;
                break;
            }
        }
 
        printf("O número divisível por 17, 18 e 20 entre 1 e 1000000 é: %d", num);
        return 0;
}
\end{lstlisting}

\subsection{Terminar iteração com \texttt{continue}}

O comando \texttt{continue} é parecido ao anterior só que, em vez de cancelar todo o ciclo de repetições, salta para a próxima iteração, cancelando a atual.

Precisa de somar todos os números inteiros entre 0 e 1000 que não são múltiplos de 2 nem de 3. Para o fazer, irá ter que saltar todas as somas em que o número atual é múltiplo de 2 e 3. Veja então uma proposta de como poderia ficar:

\begin{lstlisting}
#include <stdio.h>
 
int main() {
    int sum = 0;
    
    for (int count = 1; count <= 1000; count++) {
        if(count%2 == 0 || count%3 == 0) {
            continue;
        }
 
        sum += count;
    }
 
    printf("Soma %d", sum);
    return 0;
}
\end{lstlisting}

O que acontece é se percorrem todos os números de 1 a 1000, e se for divisível por 2 ou por 3, salto para o próximo número. Caso esta condição não se verifique, adiciono o número atual à soma.